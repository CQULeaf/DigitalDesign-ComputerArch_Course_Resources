\section{实验过程记录}

\subsection{问题1: 实现各指令码所对应的功能}
\textbf{问题描述:} 在实现过程中需要根据指令码执行相应的功能,以确保输入的指令码能够准确执行相应的算术运算。

\textbf{解决方案:} 使用 Verilog 中的 case 语句来根据指令码执行相应的操作。

\subsection{问题2: num1无符号扩展}
\textbf{问题描述:} num1由外部输入后, 需经过无符号扩展至 32 位后输入到运算器的端口B

\textbf{解决方案:} 通过在 num1 的高位添加 24 个零来实现的。具体来说,{24'b0, num1} 就是将 num1 的低 8 位放在结果的低位,然后在高位添加 24 个零。这样就实现了对 num1 进行无符号扩展,确保了它与 num2 在进行运算时具有相同的位数。

\subsection{问题3: SLT功能的实现}
\textbf{问题描述:} 实现num1与 num2 的比较,并根据比较结果设置相应的输出。

\textbf{解决方案:} 首先对 num1 进行无符号扩展,即在 num1 的高位添加 24 个零。将无符号扩展后的 num1 与 num2 进行比较。如果 num1 小于 num2,则将结果设置为 32 位全 1 的值;否则,将结果设置为 32 位全 0 的值。

\subsection{问题4: ALU约束文件完善}
\textbf{问题描述:} 给出的约束文件中存在部分端口缺失的情况,上板验证时出现错误。

\textbf{解决方案:} 补充缺失的端口到约束文件中,确保所有端口都得到正确约束,重新上板验证后无误

\subsection{问题5: 验证ALU功能}
\textbf{问题描述:} 编写仿真文件验证 ALU 的各项功能是否符合预期。

\textbf{解决方案:} 在仿真文件中设置不同的操作码 op\_tb 和操作数 num1\_tb,然后观察输出结果 ans\_tb 来验证 ALU 的各种功能是否正确。对于每种操作,将适当的操作码赋给 op\_tb 变量,并设置适当的输入操作数。每次更改操作后,等待一段时间以确保 ALU 有足够的时间执行操作。 通过观察输出结果,可以验证 ALU 模块中对应操作的功能是否正确。

\subsection{问题6: 实现4级流水线全加器}
\textbf{问题描述:} 每个级别进行 8 位加法运算,并实现流水线结构以提高计算效率。

\textbf{解决方案:} 使用四个always块进行流水线的实现,每个阶段可以在不同的时钟周期内独立计算部分8bit的结果,从而实现了流水线加法器的功能。在每个时钟周期内,每个阶段的计算结果被传递到下一个阶段,并随着时钟信号的推动完成整个加法操作。

\subsection{问题7: 实现对流水线的阻塞}
\textbf{问题描述:} 实现带有流水线暂停和刷新 4 级流水线全加器。

\textbf{解决方案:} 通过在每个阶段中加入对 stop 信号的判断逻辑,并在满足阻塞条件时暂停计算,可以实现有阻塞的流水线设计。

\subsection{问题8: 对4级流水线全加器进行行为仿真}
\textbf{问题描述:} 编写仿真文件实现对流水线功能的验证和模拟流水线的暂停和刷新。

\textbf{解决方案:} 按照时钟周期,逐步输入测试数据,观察每个阶段的计算结果是否符合预期。并在文件中要求的时刻设置暂停信号,验证流水线的暂停功能。然后恢复暂停,继续观察流水线的计算结果。