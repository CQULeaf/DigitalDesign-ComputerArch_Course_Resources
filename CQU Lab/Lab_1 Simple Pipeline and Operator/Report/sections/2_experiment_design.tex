\section{实验设计}
\subsection{ALU}\label{sub:alu}
\subsubsection{功能描述}
实现一个计算逻辑单元,对数据进行基本的操作并在数码管上直观的显示
\subsubsection{接口定义}
\begin{table}[htp]
	\caption{接口定义}\label{tab:signaldef1}
	\begin{center}
		\begin{tabular}{|l|l|l|p{6cm}|}
		\hline
		\textbf{信号名} & \textbf{方向} & \textbf{位宽} & \textbf{功能描述}\\ \hline \hline
		ans		& output	& 32-bit	& ALU的结果输出。\\ \hline
		num1	& input		& 8-bit		& 计算操作数。\\ \hline
		num2	& input		& 32-bit	& 第二个计算操作数。\\ \hline
		op		& input		& 3-bit		& 多位选择器的选择数,不同数字对应着对Input进行不同的操作。\\ \hline
		clk		& input		& 1-bit		& 时钟信号。\\ \hline
		reset	& input		& 1-bit		& 复位信号。\\ \hline
		s		& input		& 32-bit	& ALU计算结果。\\ \hline
		seg		& output	& 7-bit		& Seg7输出,七段数码管显示的数字。\\ \hline
		ans		& output	& 8-bit		& 控制哪个管子亮。\\ \hline
		sw		& input		& 3-bit		& op信号。\\ \hline
		ins		& input		& 8-bit		& Num1信号。\\ \hline
		\end{tabular}
	\end{center}
\end{table}		

\subsubsection{逻辑控制}
主要分为四个文件,ALU为核心部分,输入数据num1和num2并根据op对数据进行计算,ALU\_top为顶层文件,将所有文件功能串联起来从而实现功能,display利用时间clk并对之进行计数,从而在不同时间里让不同的管子输出alu\_result的不同位数的结果,seg7为七段数码管的控制文件根据display的din实现不同的dout
\subsection{有阻塞4级8bit全加器}\label{sub:ctl}
\subsubsection{功能描述}
将数据位数比较长的数分为多个阶段进行求解,从而提升整体运算速度
\subsubsection{接口定义}
\begin{table}[htp]
	\caption{接口定义}\label{tab:signaldef2}
	\begin{center}
		\begin{tabular}{|l|l|l|p{6cm}|}
		\hline
		\textbf{信号名} & \textbf{方向} & \textbf{位宽} & \textbf{功能描述}\\ \hline \hline
		cin\_a	& input	& 8-bit	& 要进行计算的8位数据。\\ \hline
		cin\_b	& input	& 8-bit	& 要进行计算的另一组8位数据。\\ \hline
		rst		& input	& 1-bit	& 复位信号。\\ \hline
		clk		& input	& 1-bit	& 时钟信号。\\ \hline
		stop	& input	& 1-bit	& 实现对流水线的阻塞。\\ \hline
		c\_in	& input	& 1-bit	& 模拟最开始的进位。\\ \hline
		c\_out	& output	& 1-bit	& 进位输出。\\ \hline
		sum		& output	& 8-bit	& 最终结果。\\ \hline
		\end{tabular}
	\end{center}
\end{table}
	
\subsubsection{逻辑控制}
整体分为四个阶段,每个阶段原理类似,选取阶段1进行解释,三种情况,rst为1时数据设置为0,stop为1时数据不做处理,对cin\_a和cin\_b的前两位进行操作,同时将cin\_a和b的剩下几位赋值给surA1和B1,剩下几个同理,最终可以得到8位数字的运算结果。