\section{实验内容}
\subsection{ALU设计实验}
实验要求实现以下算术运算功能,其对应的控制码及功能如下:
\begin{table}[htbp]
    \centering
    \begin{tabular}{cccc}
        \hline
         F$_{2:0}$ & 功能 & F$_{2:0}$ & 功能  \\
         \hline
         000 & A + B(Unsigned) & 100 & $\overline{A}$ \\
         001 & A - B & \textcolor{red}{\textit{\textbf{101}}} & \textcolor{red}{\textit{\textbf{SLT}}}\\
         010 & A AND B & 110 & 未使用\\
         011 & A OR B  & 111 & 未使用\\
         \hline
    \end{tabular}
    \caption{算数运算控制码及功能}
    \label{tab:opcode}
\end{table}

\textbf{实验要求:}
\begin{enumerate}
    \item 根据ALU原理图,使用Verilog语言定义ALU模块,其中输入输出端口参考实验原理,运算指令码长度为[2:0]。
    \item 仿真时B端口输入为32h'01, A端口输入参照 4.1 中表格
    % \item 内置一个32位num2(值为32h’01)作为输入到运算器端口A;
    % \item 将sw0-sw7输入到num1,经过无符号扩展 至32位后,输入到运算器的端口B;
    % \item 运算器支持“加、减、与、或、非”5种运算,需要3位(8个操作)。将sw15-sw14输入到op作为运算器的控制信号;
    \item 实现SLT功能。
    % \item 将计算32位结果s显示到七段数码管(16进制)。
    \item 验证表\ref{tab:opcode}中所有功能。
    \item 给出RTL源程序(.v文件)

\end{enumerate}

\subsection{流水线实验}

本次实验为仿真实验,设计完成后仅需进行\textbf{行为仿真}。

\textbf{实验要求:}

% 本次实验使用Vivado的Block Memory Generator模拟数据在存储器中的存取过程。实验使用单端口ROM。初始化ROM存储器中的内容,通过开关选择相应的地址,将对应的存储器中内容读出来,并通过七段数码管显示。实验原理如图~\ref{fig:memory_experiment}所示:

% \begin{figure}[htbp]
%     \centering
%     \includegraphics[width = \textwidth]{image/1_section/memory_experiment.png}
%     \caption{Caption}
%     \label{fig:memory_experiment}
% \end{figure}

% \textbf{实验要求:}
\begin{enumerate}
    \item 实现4级流水线8bit全加器,需带有流水线暂停和刷新;
    \item 模拟流水线暂停,仿真时控制10周期后暂停流水线2周期(第2级),流水线恢复流动;
    \item 模拟流水线刷新,仿真时控制15周期时流水线刷新(第3级)。

\end{enumerate}