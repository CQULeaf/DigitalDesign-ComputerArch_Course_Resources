\section{实验设计}
\subsection{数据通路模块}\label{sub:Datapath}
\subsubsection{功能描述}
该datapath模块实现了MIPS微处理器的数据通路,包括取指、译码、执行、访存和写回等五个阶段的功能。
它能够根据指令执行算术逻辑操作、控制指令的分支跳转、处理数据的前递和冒险以及控制指令的流水线执行。
\subsubsection{接口定义}
\begin{table}[htp]
	\caption{数据通路接口定义}\label{tab:Datapathdef}
	\begin{center}
		\begin{tabular}{|l|l|l|p{8cm}|}
		\hline
		\textbf{信号名} & \textbf{方向} & \textbf{宽度} & \textbf{含义}\\ \hline \hline
		clk        & 输入  & 1  & 时钟信号 \\ 
		rst        & 输入  & 1  & 异步复位信号 \\ 
		pcF        & 输出  & 32 & 取指阶段的程序计数器输出 \\ 
		instrD     & 输入  & 32 & 译码阶段的指令输入 \\ 
		regwriteD  & 输入  & 1  & 译码阶段的寄存器写使能信号 \\ 
		memtoregD  & 输入  & 1  & 译码阶段的内存到寄存器选择信号 \\ 
		branchD    & 输入  & 1  & 译码阶段的分支控制信号 \\ 
		memwriteD  & 输入  & 1  & 译码阶段的内存写使能信号 \\ 
		jumpD      & 输入  & 1  & 译码阶段的跳转控制信号 \\ 
		regdstD    & 输入  & 1  & 译码阶段的寄存器目标选择信号 \\ 
		alusrcD    & 输入  & 1  & 译码阶段的 ALU 源选择信号 \\ 
		pcsrcD      & 输入  & 1  & 译码阶段的程序计数器源选择信号 \\ 
		alucontrolID & 输入  & 3  & 译码阶段的 ALU 控制信号 \\ 
		equalID     & 输出  & 1  & 译码阶段的比较结果输出 \\ 
		op          & 输出  & 6  & 译码阶段的指令操作码输出 \\ 
		funct       & 输出  & 6  & 译码阶段的指令功能码输出 \\ 
		stallID     & 输出  & 1  & 译码阶段的流水线暂存信号 \\ 
		readdataM   & 输入  & 32 & 访存阶段的内存读数据输入 \\ 
		writedataM  & 输出  & 32 & 访存阶段的内存写数据输出 \\ 
		aluoutM     & 输出  & 32 & 访存阶段的 ALU 输出 \\ 
		memwriteM   & 输出  & 1  & 访存阶段的内存写使能信号 \\ 
		\hline
		\end{tabular}
	\end{center}
	\end{table}
\subsubsection{逻辑控制}
\begin{enumerate}
	\item \textbf{写控制}:当写使能信号 \texttt{we3} 为高时,在时钟信号 \texttt{clk} 的上升沿,将输入数据 \texttt{wd3} 写入由写地址 \texttt{wa3} 指定的寄存器。
	\item \textbf{读控制}:读操作是组合逻辑,不受时钟控制。当读地址 \texttt{ra1} 或 \texttt{ra2} 不为 0 时,分别从寄存器文件 \texttt{rf} 中读取对应地址的寄存器值输出到 \texttt{rd1} 或 \texttt{rd2};如果读地址为 0,则输出 0,这是因为寄存器 0(\texttt{\$zero})在 MIPS 架构中是硬连线为 0 的寄存器。
\end{enumerate}
\subsection{触发器(flopenrc)模块}\label{sub:flopenrc}
\subsubsection{功能描述}
一个具有异步复位和清除功能的寄存器,能够根据时钟信号、异步复位信号、清除信号和使能信号来更新其输出值。
\subsubsection{接口定义}
\begin{table}[htp]
	\caption{触发器接口定义}\label{tab:flopdef}
	\begin{center}
		\begin{tabular}{|l|l|l|p{8cm}|}
		\hline
		\textbf{信号名} & \textbf{方向} & \textbf{宽度} & \textbf{含义}\\ \hline \hline
		clk    & 输入  & 1         & 时钟信号 \\ 
		rst    & 输入  & 1         & 异步复位信号 \\ 
		en     & 输入  & 1         & 使能信号 \\ 
		clear  & 输入  & 1         & 清除信号 \\ 
		d      & 输入  & WIDTH-1:0 & 数据输入 \\ 
		q      & 输出  & WIDTH-1:0 & 寄存器输出 \\ 
		\hline
		\end{tabular}
	\end{center}
	\end{table}
\subsubsection{逻辑控制}
\begin{enumerate}
    \item \textbf{复位逻辑}:当 \texttt{rst} 信号为高电平时,无论 \texttt{en} 和 \texttt{clear} 信号的状态如何,寄存器的输出 \texttt{q} 将被重置为 0。
    \item \textbf{清除逻辑}:当 \texttt{clear} 信号为高电平时,无论 \texttt{en} 信号的状态如何,寄存器的输出 \texttt{q} 将被重置为 0。
    \item \textbf{使能逻辑}:当 \texttt{en} 信号为高电平时,寄存器的输出 \texttt{q} 将更新为输入 \texttt{d} 的值。
\end{enumerate}

\textbf{优化逻辑}:包括使用异步复位和清除信号来快速初始化寄存器状态,以及确保在时钟上升沿时执行更新操作,以避免亚稳态问题。
\subsection{冒险处理模块}\label{sub:hazard}
\subsubsection{功能描述}
用于检测和处理流水线中的数据冒险和控制冒险。
\subsubsection{接口定义}
\begin{table}[htp]
	\caption{冒险接口定义}\label{tab:hazarddef}
	\begin{center}
		\begin{tabular}{|l|l|l|p{8cm}|}
		\hline
		\textbf{信号名} & \textbf{方向} & \textbf{宽度} & \textbf{含义}\\ \hline \hline
		rsE        & 输入  & 5  & EX阶段寄存器读取源E \\ 
		rtE        & 输入  & 5  & EX阶段寄存器读取源E \\ 
		writeregM  & 输入  & 5  & MEM阶段写寄存器地址 \\ 
		writeregW  & 输入  & 5  & WB阶段写寄存器地址 \\ 
		rsD        & 输入  & 5  & DEC阶段寄存器读取源D \\ 
		rtD        & 输入  & 5  & DEC阶段寄存器读取源D \\ 
		regwriteM  & 输入  & 1  & MEM阶段寄存器写使能 \\ 
		regwriteW  & 输入  & 1  & WB阶段寄存器写使能 \\ 
		memtoregE  & 输入  & 1  & EX阶段内存到寄存器选择 \\ 
		forwardAE  & 输出  & 2  & EX阶段ALU数据前推 \\ 
		forwardBE  & 输出  & 2  & EX阶段ALU数据前推 \\ 
		stallF     & 输出  & 1  & FETCH阶段流水线暂停 \\ 
		stallID    & 输出  & 1  & DECODE阶段流水线暂停 \\ 
		\hline
		\end{tabular}
	\end{center}
	\end{table}
\subsubsection{逻辑控制}
\begin{enumerate}
    \item \textbf{数据前推逻辑}:
    \begin{itemize}
        \item \texttt{forwardAE} 和 \texttt{forwardBE}:根据 EX 阶段的寄存器读取源和 MEM/WB 阶段的写寄存器地址,确定 ALU 数据的前推方向。
        \item \texttt{forwardAD} 和 \texttt{forwardBD}:根据 DEC 阶段的寄存器读取源和 EX 阶段的写寄存器地址,确定 ALU 数据的前推方向。
    \end{itemize}

    \item \textbf{流水线暂停逻辑}:
    \begin{itemize}
        \item \texttt{stallF}:当 EX 阶段的读取源与 MEM/WB 阶段的写寄存器地址匹配时,暂停 FETCH 阶段的流水线。
        \item \texttt{stallD}:当 DECODE 阶段的读取源与 EX 阶段的写寄存器地址匹配时,暂停 DECODE 阶段的流水线。
    \end{itemize}

    \item \textbf{清除流水线逻辑}:
    \begin{itemize}
        \item \texttt{flushE}:当检测到数据冒险或控制冒险时,清除流水线,防止错误的数据流过。
    \end{itemize}
    
\end{enumerate}

\textbf{优化逻辑}:包括使用更复杂的检测机制来处理更多的冒险情况,以及确保在检测到冒险时及时清除流水线。
