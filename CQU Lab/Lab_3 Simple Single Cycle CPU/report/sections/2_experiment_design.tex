\section{实验设计}
\subsection{数据通路(datapath)}\label{sub:datapath}
\subsubsection{功能描述}
计算机中用于执行指令的路径,它包含了各种功能模块(如寄存器、ALU、存储器等)以及它们之间的连接,实现了指令的操作流程和数据的传输。
\subsubsection{接口定义}
\begin{table}[htp]
	\caption{数据通路接口定义}\label{tab:datapathdef}
	\begin{center}
		\begin{tabular}{|l|l|l|p{8cm}|}
		\hline
		\textbf{信号名} & \textbf{方向} & \textbf{宽度} & \textbf{含义}\\ \hline \hline
		clk         & in  & 1  & 时钟信号 \\ 
		rst         & in  & 1  & 复位信号 \\ 
		memstoreg   & in  & 1  & 回写的数据来自于 ALU 或者存储器 \\ 
		memwrite    & in  & 1  & 是否要书写数据存储器 \\ 
		pcsrc       & in  & 1  & 下一个 PC 是 CP+4 还是新地址 \\ 
		alusrc      & in  & 1  & 送入的是立即数的 32 位扩展还是寄存器堆读取的值 \\ 
		regdst      & in  & 1  & 写入寄存器堆的是 rt 还是 rd \\ 
		regwrite    & in  & 1  & 是否书写寄存器堆 \\ 
		jump        & in  & 1  & 跳转指令 \\ 
		alucontrol  & in  & 3  & 指令结果 \\ 
		instr       & in  & 32 & 指令码 \\ 
		readdata    & in  & 32 & 从存储器读取的数据 \\ 
		pc          & out & 32 & 程序计数器的地址 \\ 
		aluout      & out & 32 & 算术逻辑单元的计算结果 \\ 
		writedata   & out & 32 & 要写入存储器的数据 \\ 
		zero        & out & 1  & 是否为 0 \\ 
		\hline
		\end{tabular}
	\end{center}
	\end{table}
\subsubsection{逻辑控制}
Datapath将pc,mux2,pc\_inst,branch,reg,alu等结合起来,
代码的思路是构建一个能够执行MIPS指令的数据路径。
数据路径包括指令的取值、解码、执行和写回阶段。
PC用于跟踪下一条指令的地址,寄存器堆用于存储操作数和结果,ALU执行计算,多路选择器用于选择数据来源和写回的目的地。
整个数据路径由多个组合逻辑和时序逻辑组件组成,它们在时钟信号的控制下协同工作以执行指令。

\subsection{寄存器堆(Regfile)}\label{sub:regfile}
\subsubsection{功能描述}
一个小型、高速存储单元集合,用于在处理器执行指令过程中临时存储操作数和中间结果。
\subsubsection{接口定义}
\begin{table}[htp]
	\caption{寄存器堆接口定义}\label{tab:regfile}
	\begin{center}
		\begin{tabular}{|l|l|l|p{8cm}|}
		\hline
		\textbf{接口信号名} & \textbf{方向} & \textbf{宽度} & \textbf{含义}\\ \hline \hline
		clk & 输入 & 1 & 时钟信号,用于同步寄存器文件的操作 \\ 
		we3 & 输入 & 1 & 写使能信号,为1时允许写操作 \\ 
		ra1 & 输入 & 5 & 第一个读地址,用于选择第一个源寄存器 \\ 
		ra2 & 输入 & 5 & 第二个读地址,用于选择第二个源寄存器 \\ 
		wa3 & 输入 & 5 & 写地址,用于选择写回的寄存器 \\ 
		wd3 & 输入 & 32 & 写数据,将要写入寄存器的数据 \\ 
		rd1 & 输出 & 32 & 第一个读数据,来自第一个源寄存器的值 \\ 
		rd2 & 输出 & 32 & 第二个读数据,来自第二个源寄存器的值 \\ 
		\hline
		\end{tabular}
	\end{center}
	\end{table}
\subsubsection{逻辑控制}
在 \texttt{regfile} 模块的逻辑控制部分,重点控制逻辑如下:

\begin{enumerate}
    \item 写控制:当写使能信号 \texttt{we3} 为高时,在时钟信号 \texttt{clk} 的上升沿,将输入数据 \texttt{wd3} 写入由写地址 \texttt{wa3} 指定的寄存器中。

    \item 读控制:读操作是组合逻辑,不受时钟控制。当读地址 \texttt{ra1} 或 \texttt{ra2} 不为 0 时,分别从寄存器文件 \texttt{rf} 中读取对应地址的寄存器值输出到 \texttt{rd1} 或 \texttt{rd2};如果读地址为 0,则输出 0,这是因为寄存器 0(\texttt{\$zero})在 MIPS 架构中是硬连线为 0 的寄存器。
\end{enumerate}