\section{实验过程记录}

\subsection{问题1: 程序计数单元的实现}
\textbf{问题描述:} 用 PC 模块和加法器共同构成一个完整的程序计数器功能单元。

\textbf{解决方案:} PC 模块存储当前的程序计数器值并控制指令使能信号,而加法器模块用于计算下一个程序计数器值,每个时钟周期加 4。通过将这两个模块串联,实现了程序计数器在时钟信号下按照设定的增量递增的功能。在 adder 模块中实例化一个 PC 模块,可以确保在加法器模块内部的程序计数器值返回给 PC 中。

\subsection{问题2: main\_decoder 模块译码}
\textbf{问题描述:} main\_decoder 负责判断指令类型并生成相应的控制信号。输入的指令需要正确的对应输出。

\textbf{解决方案:} 查询给出的附录文件(附录 A\_coe 文件涉及指令一查表.pdf),利用 \texttt{case} 语句根据文件给出正确的各控制信号的输出。

\subsection{问题3: alu\_decoder 模块译码}
\textbf{问题描述:} alu\_decoder 负责 ALU 模块控制信号的译码,需要配合 inst 低 6 位 funct,进行 ALU 模块控制信号的译码。

\textbf{解决方案:} 同问题 2。查表后利用 \texttt{assign} 组合逻辑和判断语句给 alucontrol 赋值。

\subsection{问题4: controller 中的额外端口输出}
\textbf{问题描述:} 未存在于 controller 调用的两个模块中的 pcsrc 的设计。

\textbf{解决方案:} 虽然该实验要求中未要求该端口,但考虑到完整性最终在 controller 中设计了该端口。利用 \texttt{assign} 赋值为 branch 和 zero 变量的逻辑与操作,实现端口的设计。

\subsection{问题5: 指令存储器实现}
\textbf{问题描述:} 使用 Block Memory Generator IP 构造指令存储器。

\textbf{解决方案:} 根据指导书中的参考进行,导入给出的 coe 文件,并注意了地址深度的设置,此影响后续仿真和上板验证时的时间设置和指令循环。

\subsection{问题6: 时钟分频}
\textbf{问题描述:} 实验要求在七段数码管上自动显示指令。控制显示频率需要时钟分频。

\textbf{解决方案:} \texttt{clk\_div} 模块内部定义了一个计数器 count。没有复位信号时递增计数器 count 的值。当计数器达到一定值时,输出信号翻转一次,并将 count 清零,从而实现了时钟信号的分频。

\subsection{问题7: 仿真调试}
\textbf{问题描述:} 自行编写 Testbench 仿真文件实现仿真。

\textbf{解决方案:} 为实现仿真调试,需编写 Testbench 仿真文件并实例化 top 模块。设置时钟周期为 10ns,在每个时钟周期内输入地址 Address + 4,捕获输出信号并打印在控制台。通过运行仿真,可获得仿真图和控制台输出结果。

\subsection{问题8: 约束文件实现}
\textbf{问题描述:} 利用约束文件实现管脚配置。

\textbf{解决方案:} 根据给出的 Nexys4 板子的约束文件做出调整,时钟连接板载时钟,rst 连接板上实体按键,七段数码管显示指令后上板观察验证是否无误。